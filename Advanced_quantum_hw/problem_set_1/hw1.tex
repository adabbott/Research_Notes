\documentclass{article}
\usepackage{amsmath,mathtools}
\usepackage{graphicx}
\usepackage{braket}
\DeclareMathOperator{\sech}{sech}


\title{Molecular Quantum Mechanics Problems 1-3}
\date{1-25-2018}
\author{Adam Abbott}
\begin{document}
\maketitle
\newpage
\section{Problem 1}
\subsection{Part (a)}
First, the force due to the electric field of strength $\epsilon(t)$ can be found by considering
an electric field for a point charge which is defined by $E = \frac{F}{q}$. Therefore the force
due to the electric field for one of the charged atoms is $q\epsilon(t)$. Combined with the force
given by Hooke's law we find that for the two atoms Newton's second law gives
\[m_+ \frac{d^2 r_+}{dt^2} = -k(r_+ - r_- - r_e) + q\epsilon(t)  \]
\[m_- \frac{d^2 r_-}{dt^2} =  k(r_+ - r_- - r_e) - q\epsilon(t)  \]
%where the sign differences are due to each atom moving in differing directions under vibrational motion.
subtracting these two expressions gives
\[\frac{d^2 r_+}{dt^2} - \frac{d^2 r_-}{dt^2} = \frac{1}{m_+} (-k(r_+ - r_- - r_e) + q\epsilon(t)) + \frac{1}{m_-} (-k(r_+ - r_- - r_e) + q\epsilon(t))  \]
define the displacement $r = r_+ - r_-$ and factor the right side:
\[\frac{d^2 r}{dt^2} = (\frac{1}{m_+} + \frac{1}{m_-}) (-k(r - r_e) + q\epsilon(t)) \]
multiply $\frac{1}{m_+}$ by $\frac{m_-}{m_-}$ and $\frac{1}{m_-}$ by $\frac{m_+}{m_+}$ to obtain an expression in terms of reduced mass:
\[\frac{d^2 r}{dt^2} = \frac{1}{\mu}  (-k(r - r_e) + q\epsilon(t)) \]
and we obtain the desired equation of motion:
\[\mu \frac{d^2 r}{dt^2} = -k(r - r_e) + q\epsilon(t) \]

\subsection{Part (b)}
\[ \mu \frac{d^2 r}{dt^2} = q \epsilon(t) - k(r - r_e) \]
\[ \mu \frac{d^2 r}{dt^2} = q \epsilon_0 \cos\left[2\pi \nu (t - t_0) \right] \sech \left[2 \pi \beta (t - t_0) \right] - k(r - r_e) \]
Since $\tau = 2 \pi \nu_e t$ we have that $\frac{\tau}{\nu_e} = 2 \pi t$
and $\frac{\tau_0}{\nu_e} = 2 \pi t_0$ so that 
\[ \mu \frac{d^2 r}{dt^2} = q \epsilon_0 \cos\left[ \frac{\nu}{\nu_e}(\tau - \tau_0) \right] \sech \left[\frac{\beta}{\nu_e} (\tau - \tau_0) \right] - k(r - r_e) \]
Rearranging we get
\[ \mu \frac{d^2 r}{dt^2} + k(r - r_e)  = q \epsilon_0 \cos\left[ \frac{\nu}{\nu_e}(\tau - \tau_0) \right] \sech \left[\frac{\beta}{\nu_e} (\tau - \tau_0) \right] \]
%\[ \mu \frac{d^2 r}{dt^2} = q \epsilon(t) - k(r - r_e) \]
We can likewise substitute $\tau$ for $t$ on the left side, and, since $u = (r - r_e) /  d_0$ we have
\[ \mu \frac{d^2 (ud_0 + r_e) }{d(\tau/2\pi\nu_e)^2} + k(u d_0)  = q \epsilon_0 \cos\left[ \frac{\nu}{\nu_e}(\tau - \tau_0) \right] \sech \left[\frac{\beta}{\nu_e} (\tau - \tau_0) \right] \]
\[ (2 \pi \nu_e)^2 \mu d_0 \frac{d^2 u}{d\tau^2} + k u d_0  = q \epsilon_0 \cos\left[ \frac{\nu}{\nu_e}(\tau - \tau_0) \right] \sech \left[\frac{\beta}{\nu_e} (\tau - \tau_0) \right] \]
Since $\nu_e = \frac{\omega}{2\pi} = \sqrt{\frac{k}{\mu}}$
\[ k d_0 \frac{d^2 u}{d\tau^2} + k u d_0  = q \epsilon_0 \cos\left[ \frac{\nu}{\nu_e}(\tau - \tau_0) \right] \sech \left[\frac{\beta}{\nu_e} (\tau - \tau_0) \right] \]
\[ k d_0 \left[ \frac{d^2 u}{d\tau^2} + u \right]  = q \epsilon_0 \cos\left[ \frac{\nu}{\nu_e}(\tau - \tau_0) \right] \sech \left[\frac{\beta}{\nu_e} (\tau - \tau_0) \right] \]

$\epsilon_0$ is the maximum electric field amplitude which can displace the molecule with force $q \epsilon_0$ to displacement $d_0$. The restoring force
due to the oscillator $kd_0$ must be equal to $q \epsilon_0$, so the fraction $\frac{kd_0}{q \epsilon_0} = 1$. We are therefore left with the desired expression:
\[\left[ \frac{d^2 u}{d\tau^2} + u \right]  = \cos\left[ \frac{\nu}{\nu_e}(\tau - \tau_0) \right] \sech \left[\frac{\beta}{\nu_e} (\tau - \tau_0) \right] \]

\subsection{Part (c)}
The vibrational energy is the sum of the kinetic and potential energy $E = T + V$.
In general, the kinetic energy is $T = \frac{\rho^2}{2\mu}$ so we need to express this in terms of our reduced variables.
First, we will rewrite the kinetic energy as 
\[T = \frac{ \mu \left(\frac{dr}{dt} \right)^2 }{2} \]
Substituting in our reduced variable definitions for $u$ and $\tau$ we obtain
\[T = \frac{\mu}{2} \left( \frac{d(ud_0 + r_e)}{d(\frac{\tau}{2 \pi \nu_e})} \right)^2 \]
\[T = \frac{\mu}{2} \left( 2 \pi \nu_e d_0 \frac{du}{d\tau} \right)^2 \]
\[T = \frac{\mu}{2} (2 \pi \nu_e d_0)^2  \left(\frac{du}{d\tau}\right)^2 \]
\[T = \frac{\mu}{2} \omega^2 d_0^2  \left(\frac{du}{d\tau}\right)^2 \]
\[T = \frac{\mu}{2} \frac{k}{\mu} d_0^2  \left(\frac{du}{d\tau}\right)^2 \]
\[T = \frac{1}{2} k d_0^2  \left(\frac{du}{d\tau}\right)^2 \]
The potential energy is 
\[V = \frac{1}{2} k (r - r_e)^2  \]
\[V = \frac{1}{2} k (u d_0)^2  \]
\[V = \frac{1}{2} kd_0^2 u^2  \]
Therefore we have
\[ E = T + V\]
\[ E = \frac{1}{2} k d_0^2  \left(\frac{du}{d\tau}\right)^2 + \frac{1}{2} kd_0^2 u^2 \] 
\[ E = \frac{1}{2} k d_0^2 \left[ \left(\frac{du}{d\tau}\right)^2 + u^2 \right] \] 
\[ E = V_0 \left[ \left(\frac{du}{d\tau}\right)^2 + u^2 \right] \] 


\section{Problem 2}
\subsection{Part (a)}

Inside the sphere, where the potential is 0, the Schr{\"o}dinger equation is
\[-\frac{\hbar^2}{2m} \nabla^2 \psi = E \psi  \]
in spherical coordinates,
\[ \frac{-\hbar^2}{2m} \left( \frac{\partial^2}{\partial r^2} + \frac{2}{r} \frac{\partial}{\partial r} + \frac{1}{r^2} \left[\frac{1}{\sin \theta} \frac{\partial}{\partial \theta} (\sin \theta \frac{\partial}{\partial \theta}) + \frac{1}{\sin^2 \theta} \frac{\partial^2}{\partial \phi^2}  \right]  \right) \psi(r, \theta, \phi) = E \psi(r, \theta, \phi) \]

Following through with the separation of variables we find the wavefunction can be expressed as a product of a radial part and an angular part, where the latter are the spherical harmonics;
\[\psi(r, \theta, \phi) = R(r) Y_l^m(\theta, \phi)  \]
The angular part of the wavefunction (spherical harmonics) is the same for all spherically symmetric potentials 
For this spherically symmetric potential, we need only consider the radial part of the wave function, hence, the radial Schrodinger equation:
(Griffths 4.35)

%%%%%
\[2k \frac{dR}{dk} + k^2 \frac{\partial^2 R}{\partial k^2} + \left[k^2 - l(l+1) \right] R(r) = 0 \]
The general solution to this Sturm-Liouville differential equation is composed of a linear combination of the spherical Bessel function of the first kind $j_l(k)$ and spherical Bessel function of the second kind $y_l(k)$.
\[ R(k) = Aj_l(k) + By_l(k) \]
For the boundary conditions, the $y_l(k)$ Bessel functions blow up at the origin ($r=0$) so we realize that $B=0$.
We also must have that $R(R) = 0$, so that $j_l(kR) = 0$. Thus, $(kR)$ is a zero of the $l$th spherical Bessel function $j_l(k)$. Thus, for $n$th zero of the $l$th spherical Bessel function of the first kind, which we denote
$x_{nl}$, we have that $kR = x_{nl}$ and the allowed energies are 
\[E_{nl} = x_{nl}^2 \frac{\hbar^2}{2mR^2} \]
\subsection{Part (b)}
The overall energy is just the kinetic energy of the particle:
\[\frac{1}{2}m \braket{v^2} = E_{nl} = x_{nl}^2 \frac{\hbar^2}{2mR^2}\]
\[v_{rms} =  \sqrt{\braket{v^2}} = \sqrt{x_{nl}^2 \frac{\hbar^2}{m^2R^2}}\]
\[v_{rms} = x_{nl} \frac{\hbar}{mR}\]

\subsection{Part (c)}
For an electron in the ground quantum state ($n=1$, $l=0$), 
we have the first zero of the first spherical Bessel function of the first kind, which is just $\pi$.
Therefore, $x_{nl} = \pi$ and $R$ may be found by
\[R = \frac{\pi \hbar}{v_{rms}m_{e}} \]
20\% of the velocity of the speed of light is 59,958,491.6 m/s.
Plugging in the appropriate values of the constants we obtain $R = 0.06066$ \AA (0.11463 Bohr) .


\subsection{Part (d)}
The statistical mechanical definition of pressure is
\[P = \left(\frac{\partial E}{\partial V}\right)_T \]
The volume is
\[V = \frac{4}{3} \pi R^3 \]
\[R = (\frac{3}{4}\pi)^{1/3} V^{1/3} \]
\[E_{nl} = \pi^2 \frac{\hbar^2}{2mR^2} \]
\[E_{nl} = \pi^2 \frac{\hbar^2}{2m((\frac{3}{4}\pi)^{1/3} V^{1/3})^2}\]
Taking the derivative with respect to V we obtain
\[P = \left(\frac{\partial E}{\partial V}\right)_T = \frac{2^{4/3}\hbar^2\pi^{8/3}}{3^{5/3}mV^{5/3}}  \]
Plugging in the appropriate constants and the radius obtained from part (c) we obtain $P = 1.15 \times 10^13$ atm.


\subsection{Part (e)}
This model system demonstrates that it takes a great deal of pressure to confine a particle into a sphere with an atomic-scale radius.  The strength of the forces involved in quantum confinement, such 
as that occuring on the electrons in an atom, are astronomically large.


\section{Problem 3}
\subsection{Part (a)}
The virial theorem for a Born-Oppenheimer molecular potential energy surface gives
expressions for the electronic kinetic and potential energy as follows:
\[\braket{T} = -V(S) - S \nabla V(S) \]
\[\braket{V} = 2V(S) + S \nabla V(S) \]
Where S are some internal coordinates and V(S) is the potential evaluated at S.
Suppose we have some very distant elements in their ground states with internal coordinates $S_\infty$
and a stable molecular configuration of said elements with internal coordinates $S_{eq}$.
The gradient of the potential at these geometrical configurations will be 0.
\[\nabla V(S_\infty) = \nabla V(S_{eq}) = 0 \]
We therefore have the following relations:
\[\braket{T_{eq}} = -V(S_{eq}) \] 
\[\braket{T_{\infty}} = -V(S_{\infty}) \] 
\[\braket{V_{eq}} = 2V(S_{eq}) \] 
\[\braket{V_{\infty}} = 2V(S_{\infty}) \] 
\[2\braket{T_{eq}} = -\braket{V_{eq}} \] 
\[2\braket{T_{\infty}} = - \braket{V_{\infty}} \] 
We may then deduce that 
\[ \braket{V_{eq}} - \braket{V_{\infty}} = 2 \left[ V(S_{eq}) - V(S_{\infty}) \right] \]
and
\[ \braket{V_{eq}} - \braket{V_{\infty}} = 2 \left[ \braket{T_\infty}  - \braket{T_{eq}} \right] \]
assuming the formation of the stable molecular configuration constitutes a decrease
in the value of the BO electronic potential energy, $V(S_\infty) > V(S_{eq}$, we deduce that each quantity above must be negative,
and therefore 
\[ \braket{T_{eq}} > \braket{T_{\infty}} \]
\[ \braket{V_{eq}} < \braket{V_{\infty}} \]
Therefore the formation of the molecule is characterized by an increase in kinetic energy and a decrease in potential energy.

\subsection{Part (b)}
Use textbook

\subsection{Part (c)}
Electronic wave functions of atoms computed in the BO approximation, such as Hartree-Fock and CASSCF, are optimized with respect 
to all possible orbital variations. This is equivalent to optimizing the scaling of all the coordinates, so such wavefunctions satisfy the Virial Theorem.
Hartree-Fock theory indeed underestimates the total kinetic energy of the electrons in an atomic system.
This can be shown by first observing that the Virial Theorem tells us
\[ T = \frac{nE}{n + 2} \]
where $n$ is the order of the potential function.
So for the Hartree-Fock energy $E_{HF}$ and exact electronic energy $E_{exact}$ such that $E_{HF} > E_{exact}$, we have the corresponding kinetic energies
\[ T_{exact} = \frac{nE_{exact}}{n + 2} \]
\[ T_{HF} = \frac{nE_{HF}}{n + 2} \]
$n = -1$ since the coloumbic potential is in terms of $r^{-1}$. Therefore,
\[ T_{exact} = -E_{exact} \]
\[ T_{HF} = -E_{HF}\]
Since $E_{HF} > E_{exact}$, it follows from these expressions that  $T_{HF} < T_{exact}$.

Hartree-Fock theory assumes a mean-field potential felt by each electron, whereas in reality each electron is feeling an instantaneous and continuously adjusting repulsion from its neighbors as they zip about.

\subsection{Part (d)}
The exact total electronic kinetic energy of the nitrogen atom is the sum of its ionization potentials.
The ionization potentials in wavenumbers (cm$^{-1}$) given by C. E. Moore's \textit{Atomic Energy Levels} are
\begin{table}
\begin{tabular}{cc}
IP &  \\
1st (cm$^{-1}$)  & -117,345    \\ 
2nd (cm$^{-1}$)  & -238,846.7  \\
3rd (cm$^{-1}$)  & -382,625.5  \\
4th (cm$^{-1}$)  & -624,851    \\
5th (cm$^{-1}$)  & -789,532.9  \\
6th (cm$^{-1}$)  & -4,452,800   \\
7th (cm$^{-1}$)  & -5,379,860   \\
Total (cm$^{-1}$) & -11,985,861.1 \\
Total (E$_h$) & -54.61160181\\
\end{tabular}
\end{table}
For N$_2$ at its equilibrium geometry, the exact total electronic kinetic energy is
\[T_{N_{2}} = 2T_N + D_e \] 
\[T_{N_{2}} = 2 * -54.61160181 + 0.3639982 \]
\[T_{N_{2}} = -109.5872018 \]

The r.m.s. velocity  of the electrons in a nitrogen atom is obtained by comparing the total electronic kinetic energy to the classical kinetic energy 
\[\braket{T_N} = 7\frac{1}{2}m_e \braket{v^2} \]
The r.m.s. velocity of the electrons in the nitrogen atom is
\[\braket{v^2} = \frac{2 * 54.6116 \, \mathrm{E_h} * \frac{2625500 \, \mathrm{J/mol}} {1 \, \mathrm{hartree}}}{7 * 9.10938 \times 10^{-31} \mathrm{kg}} \frac{1 \, \mathrm{mol}}{6.022 \times 10^{23}} \]
\[v_{rms} = 8.64 \times 10^6 \mathrm{m/s}\]
The r.m.s. velocity of the electrons in diatomic nitrogen is
\[\braket{v^2} = \frac{2 * 109.5872 \, \mathrm{E_h} * \frac{2625500 \, \mathrm{J/mol}} {1 \, \mathrm{hartree}}}{14 * 9.10938 \times 10^{-31} \mathrm{kg}} \frac{1 \, \mathrm{mol}}{6.022 \times 10^{23}} \]
\[v_{rms} = 8.66 \times 10^6 \mathrm{m/s}\]
There is a slight increase in the r.m.s. velocity of the electrons in N$_2$ compared to nitrogen atom due to the formation of the bond.
RATIONALIZE?

The r.m.s. velocity of the nuclei in N$_2$, we begin with the total vibrational energy
\[E_{\mathrm{vib}} = hc\left(\omega_e (\nu + \frac{1}{2}) - \omega_e x_e  (\nu + \frac{1}{2})^2 + \dots \right) \]
for $\omega_e$(N$_2$) = 2359 cm$^{-1}$ and  $\omega_e x_e$(N$_2$) = 14.3 cm$^{-1}$ ,
\[E_{\mathrm{vib}} = 1175.93 \, \mathrm{cm}^{-1}   \]
For the quadratic potential energy associated with the vibration, the Virial Theorem requires that $\braket{T} = \braket{V}$,
\[E_{\mathrm{vib}} = 1175.93 \, \mathrm{cm}^{-1} = \braket{T} + \braket{V} = 2\braket{T}  \]
\[1175.93 \, \mathrm{cm}^{-1} = 2\braket{T} = 2 \left(\frac{1}{2} m_N \braket{v_{N_2}^2} \right)  \]
\[ \braket{v_{N_2}^2} = \frac{1175.93 \, \mathrm{cm}^{-1}  }{2} \left(\frac{1000 \, \mathrm{J/mol}}{83.59 \, \mathrm{cm}^{-1}}\right) \left(\frac{1 \, \mathrm{mol}}{14.006 \times 10^{-3} \, \mathrm{kg/mol}}\right)  \]
\[ v_{rms} = 709.84 \, \mathrm{m/s} \]
The ratio of the rms velocity of the nuclei in N$_2$ compared to the velocity of the electrons is 12,194.
That is, the electrons are moving 12,194 times faster than the nuclei. 


\end{document}
